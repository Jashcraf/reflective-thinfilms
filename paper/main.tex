\documentclass[]{spie}  %>>> use for US letter paper
%\documentclass[a4paper]{spie}  %>>> use this instead for A4 paper
%\documentclass[nocompress]{spie}  %>>> to avoid compression of citations

\renewcommand{\baselinestretch}{1.0} % Change to 1.65 for double spacing
\newcommand{\Pq}{\mathbf{P}_{q}}
 
\usepackage{amsmath,amsfonts,amssymb}
\usepackage{graphicx}
\usepackage{amsmath}
\usepackage{float}
\usepackage[colorlinks=true, allcolors=blue]{hyperref}

\title{Method of Mitigating Polarization Aberrations in Reflective Telescopes: Homogeneous Thin Films}

\author[a]{Jaren N. Ashcraft}
\author[b]{Brandon Dube}
\author[b]{Nasrat Raouf}
\author[c]{Ramya Anche}
\author[c]{Ewan S. Douglas}
\author[a,c,d]{Daewook Kim}
\author[b]{Brian Monacelli}
\author[b]{A.J.E Riggs}
\affil[a]{James C. Wyant College of Optical Sciences, University of Arizona, Meinel Building 1630 E. University Blvd., Tucson, AZ. 85721, USA}
\affil[b]{Jet Propulsion Laboratory, California Institute of Technology, 4800 Oak Grove Drive, Pasadena, CA 91109, USA}
\affil[c]{Department of Astronomy and Steward Observatory, University of Arizona, 933 N. Cherry Ave., Tucson, AZ 85721, USA}
\affil[d]{Large Binocular Telescope Observatory, University Of Arizona, 933 N. Cherry Ave. Tucson, AZ 85721, USA} 


\authorinfo{Further author information: (Send correspondence to A.A.A.)\\A.A.A.: E-mail: aaa@tbk2.edu, Telephone: 1 505 123 1234\\  B.B.A.: E-mail: bba@cmp.com, Telephone: +33 (0)1 98 76 54 32}

% Option to view page numbers
\pagestyle{empty} % change to \pagestyle{plain} for page numbers   
\setcounter{page}{301} % Set start page numbering at e.g. 301
 
\begin{document} 
\maketitle

\begin{abstract}
Next-generation ground and space telescopes aim to characterize earthlike exoplanets at closer angular separations and deeper contrasts that have been demonstrated before. One of the key factors limiting the performance of these instruments is the aberration of the wavefront due to polarization. Polarization aberrations present a unique problem in high-contrast imaging, because the orthogonally polarized parts of the incoming field are incoherent. This means that standard wavefront control techniques will be incapable of removing the aberration from both polarization states. A new method of controlling the polarization aberrations in astronomical telescopes is necessary to meet the contrast goals set by the Habitable Worlds Observatory and the Giant Segmented Mirror Telescopes. We outline a method of polarization aberration control using optimized thin-film multilayer stacks to minimize the contrast degradation in a 6 meter off-axis space telescope, and a 30 meter on-axis ground telescope. By optimizing the spatially-varying thickness of the multilayer thin film stacks, we are able to reduce the polarization leakage of a vector vortex coronagraph by an order of magnitude. We also conduct a Monte Carlo tolerance analysis to analyze the sensitivity of our optimized thin films to manufacturing errors. Polarization aberration minimized to X pico meter across Y-Z wavelengths.
\end{abstract}

% Include a list of keywords after the abstract 
\keywords{Manuscript format, template, SPIE Proceedings, LaTeX}
\section{Introduction}

\subsection{Astrophysical Goals for 2020}
Polarimetry, high-contrast imaging

\subsection{Telescope Architectures}
Next-generation telescopes for astronomy will largely take two forms. The ground-based telescopes are the Giant Segmented Mirror Telescopes (GSMTs): The European Extremely Large Telescope (ELT), The Thirty Meter Telescope (TMT), and the Giant Magellan Telescope (GMT). The GSMT's employ on-axis designs, which results in a central obscuration from the secondary mirror, and rotational symmetry of the optical system. Architechtures being considered for the future Habitable Worlds Observatory (HWO) are off-axis, meaning that the primary mirror is unobscured. This is more ideal for wavefront control, but can introduce more polarization aberration due to the higher angle of incidence. To analyze the polarization aberrations of these systems, we design two "prototype" designs to approximate the polarization aberrations of a ground-based 30 meter on-axis telescope, and a space-based 6 meter off-axis telescope.

\begin{figure}[H]
    \centering
    % \includegraphics{}
    \caption{Ray traces of the 30 meter prototype (a) and 6 meter prototype (b).}
    \label{fig:raytraces}
\end{figure}

\begin{table}[H]
    \centering
    \begin{tabular}{c c c c c}
    \hline
        Surface & RoC & Conic Constant & Tilt & Diameter  \\
    \hline
        1* & & & & \\
        2 & & & & \\
        3 & & & & \\
    \hline
    \hline
    \end{tabular}
    \caption{Caption}
    \label{tab:ground_prescription}
\end{table}

\begin{table}[H]
    \centering
    \begin{tabular}{c c c c c}
    \hline
        Surface & RoC & Conic Constant & Tilt & Diameter  \\
    \hline
        1* & & & & \\
        2 & & & & \\
        3 & & & & \\
    \hline
    \hline
    \end{tabular}
    \caption{Caption}
    \label{tab:space_prescription}
\end{table}

\subsection{Polarization Aberrations}
Breckenridge, McGuire, Chipman
\section{Methods}
\label{sec:methods}

\subsection{Thin film characteristic matrix}
The Fresnel equations are sufficient for describing uncoated optical surfaces, but rarely is this the case. Most optical surfaces have thin dielectric coatings to enhance the transmitted or reflected signal. These coatings can be inhomogeneous, which can cause further polarization aberration. To accurately capture these effects, we need a method of computing an effective Fresnel reflection coefficient. This is commonly done (assuming the coatings are isotropic) by computing the characteristic matrix of the dielectric stack\cite{BYU,Chipman}. The derivation for these is quite involved so it will not be recreated here, but the full derivation and set of equations is shown in the free online textbook by Peatross and Ware\cite{BYU}. For p-polarized light, the characteristic matrix $A^{(p)}$ is given by the following relation.

\begin{equation}
    A^{(p)} = \frac{1}{2n_{o}cos(\theta_{o})}
    \begin{pmatrix}
        n_{o} & cos(\theta_{o}) \\
        n_{o} & -cos(\theta_{o}) \\
    \end{pmatrix}
    \prod_{j=1}^{N} M_{j}^{(p)}
    \begin{pmatrix}
        cos(\theta_{N+1}) & 0 \\
        n_{N+1} & 0 \\
    \end{pmatrix}
\end{equation}

Where $M_{j}^{(p)}$ is the j-th matrix of a layer in the thin film coating described by
\begin{equation}
    M_{j}^{(p)} = 
    \begin{pmatrix}
        cos(\beta_{j}) & -isin(\beta_{j})cos(\theta_{j})/n_{j} \\
        -in_{j}sin(\beta_{j})/cos(\theta_{j}) &  cos(\beta_{j})  \\
    \end{pmatrix}
\end{equation}

And $B_{j}$ is the optical path length along the propagation direction in the dielectric film defined by the path length from the first interface.

\begin{equation}
    \beta_{j} = 
    \begin{cases}
        0 & j = 0 \\
        k_{j}d_{j}cos(\theta_{j}) & 1 \leq j \leq N \\
    \end{cases}
\end{equation}

Similarly for the s-polarized characteristic matrix, we have

\begin{equation}
    A^{(s)} = \frac{1}{2n_{o}cos(\theta_{o})}
    \begin{pmatrix}
        n_{o}cos(\theta_{o}) & 1 \\
        n_{o}cos(\theta_{o}) & -1 \\
    \end{pmatrix}
    \prod_{j=1}^{N} M_{j}^{(s)}
    \begin{pmatrix}
        1 & 0 \\
        n_{N+1}cos(\theta_{N+1}) & 0 \\
    \end{pmatrix}
\end{equation}

Where $M_{j}^{(s)}$ is

\begin{equation}
    M_{j}^{(s)} = 
    \begin{pmatrix}
        cos(\beta_{j}) & -isin(\beta_{j})/n_{j}cos(\theta_{j}) \\
        -in_{j}sin(\beta_{j})cos(\theta_{j}) &  cos(\beta_{j})  \\
    \end{pmatrix}
\end{equation}

The effective Fresnel transmission and reflection coefficients are then derived simply in terms of the characteristic matrices.

\begin{equation}
    t_{p}^{tot} = \frac{1}{a_{00}^{(p)}}
\end{equation}

\begin{equation}
    r_{p}^{tot} = \frac{a_{10}^{(p)}}{a_{00}^{(p)}}
\end{equation}

\begin{equation}
    t_{s}^{tot} = \frac{1}{a_{00}^{(s)}}
\end{equation}

\begin{equation}
    r_{s}^{tot} = \frac{a_{10}^{(s)}}{a_{00}^{(s)}}
\end{equation}

These physics were implemented in the open-source polarization ray tracing package, Poke.

\subsection{Poke: Integrating ray and diffraction models}
Poke is an open-source Python package that has been used for polarization ray tracing simulations of the GSMT's. It features highly vectorized PRT calculations and for this investigation we added the ability to use PRT with multilayer thin film stacks. The accelerated computing is useful for running optimization cycles rapidly.

\begin{figure}
    \centering
    % \includegraphics{}
    \caption{Runtime to compute Jones pupil v.s. number of rays traced}
    \label{fig:my_label}
\end{figure}

\subsection{Constant raypaths, variable films}
Keep the orthogonal transformations the same, change the fresnel coefficients

\subsection{Telescope Architectures for Design Study}
6 meter space telescope, 30m ground telescope. These don't have to be official, but can be simmilar enough in geometry to be relevant.

\subsection{Optimization methodology}
Polarization ray tracing is conducted via a matrix product of the constituent PRT matrices to produce the system PRT matrix ($\mathbf{P}_{tot}$ in equation \ref{eq:prt_prod})

\begin{equation}
    \mathbf{P}_{tot} = \prod_{i=0}^{Q} \mathbf{P}_{i}
    \label{eq:prt_prod}
\end{equation}

The $i$-th PRT matrix is simply a product of two orthogonal transformation matrices ($\mathbf{O}$) with a diagonal jones matrix.

\begin{equation}
    \mathbf{P}_{i} = \mathbf{O}_{out,i} 
    \begin{pmatrix}
        \Lambda_{s} & 0 & 0 \\
        0 & \Lambda_{p} & 0 \\
        0 & 0 & 1 \\
    \end{pmatrix}
    \mathbf{O}_{in,i}^{-1}
\end{equation}

Where $\Lambda_{s},\Lambda_{p}$ are the fresnel coefficients for s- and p- polarizations, respectively. The matrices for orthogonal transformation are determined by the wave vector $\mathbf{k}$ and corresponding eigenpolarizations of the local surface interaction $\mathbf{s}$ and $\mathbf{p}$.

\begin{equation}
    \mathbf{O} = 
    \begin{pmatrix}
        s_{x} & s_{y} & s_{z} \\
        p_{x} & p_{y} & p_{z} \\
        k_{x} & k_{y} & k_{z} \\
    \end{pmatrix}
\end{equation}

The orthogonal transformation matrices are uniquely determined by the ray paths through the optical system. In the limit of interactions with thin film optical filters on entirely reflective optical systems, the raypaths through these optical systems do not change as a function of the film thickness or wavelength. This means that the orthogonal transformation matrices need only be computed once per optimization procedure, and the only operation that needs to be updated per optimization iteration is the computation of the Jones pupil.

The cost function we define is a simple summation over the ray index ($i$) and wavelengths ($\lambda$) of the polarization aberrations normalized by the total number of pixels ($N$). 

\begin{equation}
    \mathcal{C} = \sum_{\lambda = 600nm}^{800nm} \sum_{i=0}^{N} \frac{\delta_{i}^{2}}{N} + \frac{\mathcal{D}_{i}^{2}}{N} + \frac{(1-\mathcal{R}_{i})^{2}}{N}
\end{equation}






\section{Results}

\subsection{Jones Pupils before optimization}
We begin by comparing the Jones pupils of a bare aluminum coating v.s. a protected silver coating for the two prototypes.

\subsection{Diattenuation, Retardance pupils v.s. Spectrum}

\include{s4_Discussion}

\acknowledgments % equivalent to \section*{ACKNOWLEDGMENTS}       
 
This unnumbered section is used to identify those who have aided the authors in understanding or accomplishing the work presented and to acknowledge sources of funding.  

% References
%\bibliography{report} % bibliography data in report.bib
\bibliographystyle{spiebib} % makes bibtex use spiebib.bst

\end{document} 
